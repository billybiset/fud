\chapter{Conclusiones}
Como resultado de este trabajo se ha obtenido una nueva capa para el framework \fud, de nombre \combeng. La misma ha cumplido con todas las
expectativas y ha cubierto cada uno de los requerimientos planteados desde un comienzo. Si bien s\'olo se ha desarrollado una aplicaci\'on relevante,
y algunas otras como prueba, se puede concluir que el motor combinatorio se encuentra apto para resolver cualquier problema que requiera de
la combinaci\'on de cualquier tipo de elementos. Aquellas aplicaciones que hacen un uso correcto del framework \fud \ y, por ende, de la nueva capa,
no deber\'an preocuparse por asuntos relacionados a la programaci\'on paralela. Esto permite realizar el procesamiento de las
 combinaciones, muchas veces de gran costo computacional, de manera distribuida y sin la necesidad de tener grandes conocimientos en el \'area. 

En cuanto a la aplicaci\'on \rnaffe, se han obtenido numerosos y valiosos datos que hasta el momento no eran conocidos. Tanto la
aplicaci\'on como los datos recolectados recibieron buenas cr\'iticas tras ser presentados en el 2do Congreso Argentino de
Bioinform\'atica y Biolog\'ia Computacional. La informaci\'on obtenida sirvi\'o para dar un abordaje 
inicial a como var\'ia la energ\'ia libre a medida que se avanza en una terapia antirretroviral. Debido a la falta de recursos y tiempo,
 la aplicaci\'on no pudo ser ejecutada en su totalidad, por lo que los datos que fueron recopilados pertenecen al 60\% del \'arbol total. 
No obstante, la informaci\'on adquirida fue suficiente para obtener ciertas estad\'isticas y tendencias interesantes.
\newpage
\section{Trabajo a Futuro}
A continuaci\'on se muestran las tareas que quedan pendientes en este trabajo. Las mismas se encuentran agrupadas seg\'un pertenezcan a 
la aplicaci\'on \emph{RNAFoldingFE} o a la nueva capa acoplada a \fud (\combeng).

\subsection{Trabajo Futuro Para la Aplicaci\'on \emph{RNAFoldingFE}}
\begin{itemize}
 \item Optimizar la implementaci\'on de la FSM que define la manera en que los antirretrovirales son combinados.
 \item Implementar pol\'iticas de poda para achicar el espacio de b\'usqueda.
 \item Implementar la funcionalidades necesarias para hacer que la aplicaci\'on sea independiente de la arquitectura.
 \item Implementar una interfaz gr\'afica que permita visualizar los resultados intuitivamente.
% \item Proveer manejo de probabilidades ya que en ning\'un momento se contemplan cuales son las mutaciones m\'as probables, directamente se toma la de menor distancia.
\end{itemize}

\subsection{Trabajo futuro para \combeng}
\begin{itemize}
  \item Implementar nuevas pol\'iticas de combinaci\'on.
\end{itemize}
