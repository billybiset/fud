\chapter{Trabajos futuros}
\label{chap:future_work}

Tal vez la parte más importante sobre \rc{} es lo que se haga con ella, es por eso que a continuación se presentará una lista con proyectos
a futuro, los cuáles serán montados sobre la capa desarrollada. Además, existen funcionalidades y/o extensiones de este trabajo que serán
dejadas como futuras tareas de la fundación. He aquí algunos trabajos interesantes por llevar a cabo:
\begin{itemize}
    \item   Implementar proyectos bioinformáticos. Un caso concreto es \textit{backbones-generator}, este proyecto pertenece a \fude{} y
            tiene como objetivo construir una base de datos de esqueletos (\textit{backbones} o cadenas proteicas principales) de un tamaño
            dado a partir de una combinatoria de ángulos y restricciones. Se encuentra implementado secuencialmente, y se desea construir la
            versión \rc{} del mismo problema.
    \item   Adaptar esta capa para que proyectos \textbf{\textit{BOINC}} funcionen sobre ella. Lo que se debe hacer es distribuir los
            functores recursivos, como unidades de trabajo, entre los nodos conectados de la forma en que este middleware trabaja. El uso
            de \textbf{\textit{BOINC}} permitiría a \fude{} obtener la potencia de procesamiento de miles de CPUs en proyectos científicos.
    \item   Construir nuevas políticas de distribución para un balanceo de carga específico sobre determinadas aplicaciones o para
            definir un balanceo genérico con buen rendimiento sobre un grupo de problemas afines. En muchos casos, una \textit{buena}
            distribución depende exclusivamente del problema que deseamos resolver, es por esto que se pueden extender políticas para
            una necesidad concreta. Por el contrario, también es posible crear mecanismos de balanceo de carga que ataquen a un conjunto de
            aplicaciones, y mantengan una buena performance en todos los casos.
\end{itemize}
