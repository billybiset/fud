\chapter{Conclusión}
\label{chap:conclusion}

Con respecto a \rc{} se ha obtenido una abstracción como una nueva capa del framework \fud{} que agrupa las soluciones más recurrentes a 
problemas bioinformáticos que existen en \fude. Más específicamente, se logró definir una librería que tome como entrada la solución
recursiva a cualquier problema con subproblemas sin superposición teniendo como resultado una versión distribuida de la misma.
El trabajo realizado ha arrojado resultados de performance positivos tanto con la aplicación trivial presentada como así también con
\textit{RNAFoldingFreeEnergy} sobre el motor combinatorio \textit{CombEng}.

En relación a la refactorización de \fud{}, se obtuvo una nueva versión del framework, que cumpliendo con sus principios básicos y
manteniendo su eficiencia, permite realizar interacciones entre los clientes y el servidor. Ahora, admite un conjunto más amplio de
posibles problemas a resolver.

Este proyecto, aporta una herramienta fundamental para la futura resolución de problemas que afronte \fude{} para obtener resultados
concretos a corto plazo. Ambos resultados han cumplido con todas las expectativas y han cubierto cada uno de los objetivos
planteados desde el inicio de este proyecto.     
