% $Id$

\documentclass[a4paper,12pt]{article}
\usepackage[left=2.5cm,top=3.5cm,right=2.5cm,bottom=4cm]{geometry}

\usepackage[utf8]{inputenc}
\usepackage[spanish]{babel}


\def\rc {\textbf{\textit{recabs}}}
\def\fud {\textbf{\textit{FuD}}}
\def\fude {\textbf{\textit{FuDePAN}}}


\title{Propuesta de Trabajo Final Para la Carrera de Licenciatura en Ciencias de la Computación}
\author{Mariano Bessone y Emanuel Bringas}
\date{\today}

\begin{document}
\maketitle

    \begin{center}
        \begin{large}
            \rc{}
        \end{large}
    \end{center}
    \begin{description}
    \item[Tema:] Abstracción de distribución de procesos recursivos implementada sobre \fud{}. %Desarrollo de una capa de
    \item[Director:] Guillermo Biset.
    \item[Co-Director:] Daniel Gutson. 
    \item[Alumnos:] Bessone, Mariano José - Bringas, Emanuel César.
    \end{description}

    \section{Descripción del problema}

    Este proyecto nace como una sugerencia de \fude{}\footnote{Fundación para el desarrollo de la programación en Ácidos Nucleicos. http://www.fudepan.org.ar/ }
    , donde se le presentan un gran número de problemas que son resueltos recursivamente y poseen muchos factores de implementación
    en común. La idea es proveer un mecanismo de distribución utilizando el framework \fud{}\footnote{\fude{} Ubiquitous Distribution\cite{clus09}.
    http://code.google.com/p/fud/} que abstraiga la funcionalidad otorgada por implementaciones recursivas, abarcando un basto número
    de problemas bioinformáticos.\\

    La meta de este proyecto es identificar abstracciones genéricas que se amolden a una estructura común de las soluciones recursivas a
    resolver y al mismo tiempo proveer una solución algorítmicamente eficiente para dichas abstracciones. Esta abstracción abarcará sólo
    los algoritmos recursivos en los cuales haya independencia de datos entre los pasos de la recursión.\\

    Se espera que esta capa facilite el desarrollo de soluciones para cualquier problema de este tipo mediante la implementación de las 
    interfaces identificadas y definidas en este proyecto. Se piensa lograr que el trabajo sea distribuible, para lo 
    cual se decide aprovechar las características del framework FuD, ya que permitirá obtener mayor provecho de los recursos disponibles 
    para resolver problemas de alta complejidad computacional.\\

    Para balencear el trabajo sobre clientes disponibles se usará un modelo colaborativo para cómputos de altas prestaciones (HPC) que 
    permite distribuir programas de naturaleza recursiva\cite{LauGar09}.
 
    \section{Recursos}
    Para la obtención del material bibliográfico se cuenta con la biblioteca de la Universidad Nacional de Río Cuarto en conjunto con la 
    amplia cantidad de información que se puede encontrar en Internet y la bibliografía disponible en el Departamento de Computación mediante 
    suscripciones a publicaciones científicas.\\

    La redacción en formato digital de la tesis se llevará a cabo mediante la utilización del procesador de texto \LaTeXe. El sistema 
    operativo a utilizar será \texttt{GNU/Linux} y la herramienta final deberá ser software libre (bajo licencia GPL , versión 3 o superior).\\


    \section{Resultados Esperados}

    Usando esta capa de abstracción, se espera que la implementación de soluciones recursivas de un problema particular se realize de manera
    sencilla. Se presentarán interfaces claras para la interacción con el usuario, simplificando las tareas de diseño e implementación.\\

    Como esta capa estará montada sobre FuD, la idea es obtener una versión distribuible de la implementación sin que el usuario tenga que 
    preocuparse por lograr esta funcionalidad. De esta manera, se pueden aprovechar los recursos disponibles en proyectos de alto
    costo computacional para lograr un mejor rendimiento en cuanto a tiempo de ejecución.

    \nocite{fud09}
    \nocite{oosc}
    \nocite{cpluplus}
    \nocite{uml}

    \newpage
    \bibliographystyle{ieeetr}
    \bibliography{biblio}
    

\end{document}
