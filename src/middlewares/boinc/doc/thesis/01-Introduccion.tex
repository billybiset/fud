\chapter{Introducción}
\label{chapter:introduccion}

Hoy en día son cada vez más las entidades sin fines de lucro que realizan determinadas investigaciones científicas con el objetivo de aportar un gran avance o descubrimiento en distintas áreas tales como la medicina, Astronomía, Física, Química, etc. Tal es el caso de la organización sin fines de lucro FuDePAN\footnote{\url{http://www.fudepan.org.ar/}} (Fundación para el Desarrollo de la Programación en Ácidos Nucleicos) la cual se dedica a la investigación y desarrollo en bioinformática aplicada a problemas biológicos asociados a la salud humana. En esta fundación utilizan el cálculo computacional para hacer simulaciones sobre cómo determinados virus y enfermedades, tales como el HIV o el virus Junín, actúan en el cuerpo humano con el fin de mejorar los tratamientos y las vacunas contra los mismos.

Los problemas que generalmente son enfrentados en FuDePAN son de alta complejidad por lo que se requiere un gran poder computacional para poder resolverlos en el menor lapso de tiempo posible. Es por eso que se desarrolló el framework FuD\footnote{\url{http://code.google.com/p/fud/}} (Ubiquitous Distribution Platform) para la distribución automática de trabajos en nodos de procesamiento, el cual permite obtener soluciones paralelizadas a partir de proyectos secuenciales realizando una simple reimplementación. Mediante el uso de este framework se lograron obtener resultados confiables en lapsos de tiempo relativamente cortos.

En sus orígenes, la capa de distribución de FuD sólo fue implementada con la librería de E/S asincrónica de Boost\footnote{\url{http://www.boost.org/}} (o Boost::asio) con el objetivo de poder realizar computación de alto rendimiento. Sin embargo, para poder aprovechar al máximo dicha implementación se requería de un tipo de súper-computadora la cual era muy costosa. Es por ello que surge la motivación de poder implementar una nueva capa de distribución de FuD con el middleware para la computación voluntaria BOINC, con el fin de poder obtener un mayor poder de procesamiento a un costo significativamente menor gracias a personas voluntarias que donan los recursos ocioso de sus computadoras personales para realizar computación científica.

El objetivo de este proyecto fue poder acoplar el funcionamiento de la capa de distribución del framework FuD con la plataforma BOINC para permitirle a los usuarios de FuD contar con la posibilidad de ejecutar sus aplicaciones sobre un proyecto para la computación voluntaria. De esta manera, FuD cuenta con una variante para realizar computación de alto rendimiento y otra para la computación voluntaria.

Este documento ofrece una visión general de las tareas de investigación realizadas, del diseño, de algunos detalles de implementación y de algunas pruebas realizadas. Se muestran distintos ejemplos que facilitaron la comprensión de las tareas mencionadas anteriormente, como así también los resultados de ejecutar ciertas aplicaciones compiladas con esta nueva implementación de FuD.

El informe se divide en 5 partes. La parte  “Preliminar” incluye: una introducción a este trabajo (\ref{chapter:introduccion}), un capítulo con el marco teórico (\ref{chapter:marco:teorico}) para facilitar la lectura de este documento y otro capítulo con la metodología de trabajo (\ref{chapter:metodologia}) utilizada para desarrollar el proyecto.
	 	 	
La parte “Capa de distribución FuD-BOINC” contiene todo lo relacionado sobre este proyecto \ref{chapter:sobre:fud:boinc}. Se enuncia el problema y el enfoque de solución abordado para luego ofrecerle al lector detalles de diseño (\ref{chapter:diseno}) e implementación (\ref{chapter:implementacion}), como así también de las pruebas realizadas (\ref{chapter:pruebas}) y los resultados obtenidos (\ref{chapter:resultados}).

La parte “Conclusión” presenta la conclusión (\ref{chapter:conclusion}) que se pudo obtener luego de realizar este proyecto y los trabajos a futuro (\ref{chapter:future:work}) relacionados al mismo.

La parte “Bibliografía”, como bien indica su nombre, muestra la bibliografía (\ref{biblio}) utilizada, y finalmente, la parte “Apéndices” muestra un reporte completo de métricas de código (\ref{chapter:FuD-BOINC:metrics_report}).